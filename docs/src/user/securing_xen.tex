\chapter{Securing Xen}

This chapter describes how to secure a Xen system. It describes a number
of scenarios and provides a corresponding set of best practices. It
begins with a section devoted to understanding the security implications
of a Xen system.


\section{Xen Security Considerations}

When deploying a Xen system, one must be sure to secure the management
domain (Domain-0) as much as possible. If the management domain is
compromised, all other domains are also vulnerable. The following are a
set of best practices for Domain-0:

\begin{enumerate}
\item \textbf{Run the smallest number of necessary services.} The less
  things that are present in a management partition, the better.
  Remember, a service running as root in the management domain has full
  access to all other domains on the system.
\item \textbf{Use a firewall to restrict the traffic to the management
    domain.} A firewall with default-reject rules will help prevent
  attacks on the management domain.
\item \textbf{Do not allow users to access Domain-0.} The Linux kernel
  has been known to have local-user root exploits. If you allow normal
  users to access Domain-0 (even as unprivileged users) you run the risk
  of a kernel exploit making all of your domains vulnerable.
\end{enumerate}

\section{Security Scenarios}


\subsection{The Isolated Management Network}

In this scenario, each node has two network cards in the cluster. One
network card is connected to the outside world and one network card is a
physically isolated management network specifically for Xen instances to
use.

As long as all of the management partitions are trusted equally, this is
the most secure scenario. No additional configuration is needed other
than forcing Xend to bind to the management interface for relocation.

\textbf{FIXME:} What is the option to allow for this?


\subsection{A Subnet Behind a Firewall}

In this scenario, each node has only one network card but the entire
cluster sits behind a firewall. This firewall should do at least the
following:

\begin{enumerate}
\item Prevent IP spoofing from outside of the subnet.
\item Prevent access to the relocation port of any of the nodes in the
  cluster except from within the cluster.
\end{enumerate}

The following iptables rules can be used on each node to prevent
migrations to that node from outside the subnet assuming the main
firewall does not do this for you:

\begin{verbatim}
# this command disables all access to the Xen relocation
# port:
iptables -A INPUT -p tcp --destination-port 8002 -j REJECT

# this command enables Xen relocations only from the specific
# subnet:
iptables -I INPUT -p tcp -{}-source 192.168.1.1/8 \
    --destination-port 8002 -j ACCEPT
\end{verbatim}

\subsection{Nodes on an Untrusted Subnet}

Migration on an untrusted subnet is not safe in current versions of Xen.
It may be possible to perform migrations through a secure tunnel via an
VPN or SSH. The only safe option in the absence of a secure tunnel is to
disable migration completely. The easiest way to do this is with
iptables:

\begin{verbatim}
# this command disables all access to the Xen relocation port
iptables -A INPUT -p tcp -{}-destination-port 8002 -j REJECT
\end{verbatim}
