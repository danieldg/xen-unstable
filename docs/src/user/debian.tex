\chapter{Installing Xen / XenLinux on Debian}

The Debian project provides a tool called \path{debootstrap} which
allows a base Debian system to be installed into a filesystem without
requiring the host system to have any Debian-specific software (such
as \path{apt}).

Here's some info how to install Debian 3.1 (Sarge) for an unprivileged
Xen domain:

\begin{enumerate}

\item Set up Xen and test that it's working, as described earlier in
  this manual.

\item Create disk images for rootfs and swap. Alternatively, you might
  create dedicated partitions, LVM logical volumes, etc.\ if that
  suits your setup.
\begin{verbatim}
dd if=/dev/zero of=/path/diskimage bs=1024k count=size_in_mbytes
dd if=/dev/zero of=/path/swapimage bs=1024k count=size_in_mbytes
\end{verbatim}

  If you're going to use this filesystem / disk image only as a
  `template' for other vm disk images, something like 300 MB should be
  enough. (of course it depends what kind of packages you are planning
  to install to the template)

\item Create the filesystem and initialise the swap image
\begin{verbatim}
mkfs.ext3 /path/diskimage
mkswap /path/swapimage
\end{verbatim}

\item Mount the disk image for installation
\begin{verbatim}
mount -o loop /path/diskimage /mnt/disk
\end{verbatim}

\item Install \path{debootstrap}. Make sure you have debootstrap
  installed on the host.  If you are running Debian Sarge (3.1 /
  testing) or unstable you can install it by running \path{apt-get
    install debootstrap}.  Otherwise, it can be downloaded from the
  Debian project website.

\item Install Debian base to the disk image:
\begin{verbatim}
debootstrap --arch i386 sarge /mnt/disk  \
            http://ftp.<countrycode>.debian.org/debian
\end{verbatim}

  You can use any other Debian http/ftp mirror you want.

\item When debootstrap completes successfully, modify settings:
\begin{verbatim}
chroot /mnt/disk /bin/bash
\end{verbatim}

Edit the following files using vi or nano and make needed changes:
\begin{verbatim}
/etc/hostname
/etc/hosts
/etc/resolv.conf
/etc/network/interfaces
/etc/networks
\end{verbatim}

Set up access to the services, edit:
\begin{verbatim}
/etc/hosts.deny
/etc/hosts.allow
/etc/inetd.conf
\end{verbatim}

Add Debian mirror to:   
\begin{verbatim}
/etc/apt/sources.list
\end{verbatim}

Create fstab like this:
\begin{verbatim}
/dev/sda1       /       ext3    errors=remount-ro       0       1
/dev/sda2       none    swap    sw                      0       0
proc            /proc   proc    defaults                0       0
\end{verbatim}

Logout

\item Unmount the disk image
\begin{verbatim}
umount /mnt/disk
\end{verbatim}

\item Create Xen 2.0 configuration file for the new domain. You can
  use the example-configurations coming with Xen as a template.

  Make sure you have the following set up:
\begin{verbatim}
disk = [ 'file:/path/diskimage,sda1,w', 'file:/path/swapimage,sda2,w' ]
root = "/dev/sda1 ro"
\end{verbatim}

\item Start the new domain
\begin{verbatim}
xm create -f domain_config_file
\end{verbatim}

Check that the new domain is running:
\begin{verbatim}
xm list
\end{verbatim}

\item Attach to the console of the new domain.  You should see
  something like this when starting the new domain:

\begin{verbatim}
Started domain testdomain2, console on port 9626
\end{verbatim}
        
  There you can see the ID of the console: 26. You can also list the
  consoles with \path{xm consoles} (ID is the last two digits of the
  port number.)

  Attach to the console:

\begin{verbatim}
xm console 26
\end{verbatim}

  or by telnetting to the port 9626 of localhost (the xm console
  program works better).

\item Log in and run base-config

  As a default there's no password for the root.

  Check that everything looks OK, and the system started without
  errors.  Check that the swap is active, and the network settings are
  correct.

  Run \path{/usr/sbin/base-config} to set up the Debian settings.

  Set up the password for root using passwd.

\item Done. You can exit the console by pressing {\path{Ctrl + ]}}

\end{enumerate}


If you need to create new domains, you can just copy the contents of
the `template'-image to the new disk images, either by mounting the
template and the new image, and using \path{cp -a} or \path{tar} or by
simply copying the image file.  Once this is done, modify the
image-specific settings (hostname, network settings, etc).
