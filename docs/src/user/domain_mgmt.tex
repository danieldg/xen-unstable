\chapter{Domain Management Tools}

This chapter summarises the tools available to manage running domains.


\section{\Xend\ }
\label{s:xend}

The Xen Daemon (\Xend) (node control daemon) performs system management
functions related to virtual machines. It forms a central point of
control for a machine and can be controlled using an HTTP-based
protocol. \Xend\ must be running in order to start and manage virtual
machines.

\Xend\ must be run as root because it needs access to privileged system
management functions. A small set of commands may be issued on the
\xend\ command line:

\begin{tabular}{ll}
  \verb!# xend start! & start \xend, if not already running \\
  \verb!# xend stop!  & stop \xend\ if already running       \\
  \verb!# xend restart! & restart \xend\ if running, otherwise start it \\
  % \verb!# xend trace_start! & start \xend, with very detailed debug logging \\
  \verb!# xend status! & indicates \xend\ status by its return code
\end{tabular}

A SysV init script called {\tt xend} is provided to start \xend\ at boot
time. {\tt make install} installs this script in \path{/etc/init.d}. To
enable it, you have to make symbolic links in the appropriate runlevel
directories or use the {\tt chkconfig} tool, where available.

Once \xend\ is running, more sophisticated administration can be done
using the xm tool (see Section~\ref{s:xm}) and the experimental Xensv
web interface (see Section~\ref{s:xensv}).

As \xend\ runs, events will be logged to \path{/var/log/xend.log} and,
if the migration assistant daemon (\path{xfrd}) has been started,
\path{/var/log/xfrd.log}. These may be of use for troubleshooting
problems.

\section{Xm}
\label{s:xm}

Command line management tasks are also performed using the \path{xm}
tool. For online help for the commands available, type:

\begin{quote}
\begin{verbatim}
# xm help
\end{verbatim}
\end{quote}

You can also type \path{xm help $<$command$>$} for more information on a
given command.

The xm tool is the primary tool for managing Xen from the console. The
general format of an xm command line is:

\begin{verbatim}
# xm command [switches] [arguments] [variables]
\end{verbatim}

The available \emph{switches} and \emph{arguments} are dependent on the
\emph{command} chosen. The \emph{variables} may be set using
declarations of the form {\tt variable=value} and command line
declarations override any of the values in the configuration file being
used, including the standard variables described above and any custom
variables (for instance, the \path{xmdefconfig} file uses a {\tt vmid}
variable).

The available commands are as follows:

\begin{description}
\item[mem-set] Request a domain to adjust its memory footprint.
\item[create] Create a new domain.
\item[destroy] Kill a domain immediately.
\item[list] List running domains.
\item[shutdown] Ask a domain to shutdown.
\item[dmesg] Fetch the Xen (not Linux!) boot output.
\item[consoles] Lists the available consoles.
\item[console] Connect to the console for a domain.
\item[help] Get help on xm commands.
\item[save] Suspend a domain to disk.
\item[restore] Restore a domain from disk.
\item[pause] Pause a domain's execution.
\item[unpause] Un-pause a domain.
\item[pincpu] Pin a domain to a CPU.
\item[bvt] Set BVT scheduler parameters for a domain.
\item[bvt\_ctxallow] Set the BVT context switching allowance for the
  system.
\item[atropos] Set the atropos parameters for a domain.
\item[rrobin] Set the round robin time slice for the system.
\item[info] Get information about the Xen host.
\item[call] Call a \xend\ HTTP API function directly.
\end{description}

\subsection{Basic Management Commands}

The most important \path{xm} commands are:
\begin{quote}
  \verb_# xm list_: Lists all domains running.\\
  \verb_# xm consoles_: Gives information about the domain consoles.\\
  \verb_# xm console_: Opens a console to a domain (e.g.\
  \verb_# xm console myVM_)
\end{quote}

\subsection{\tt xm list}

The output of \path{xm list} is in rows of the following format:
\begin{center} {\tt name domid memory cpu state cputime console}
\end{center}

\begin{quote}
  \begin{description}
  \item[name] The descriptive name of the virtual machine.
  \item[domid] The number of the domain ID this virtual machine is
    running in.
  \item[memory] Memory size in megabytes.
  \item[cpu] The CPU this domain is running on.
  \item[state] Domain state consists of 5 fields:
    \begin{description}
    \item[r] running
    \item[b] blocked
    \item[p] paused
    \item[s] shutdown
    \item[c] crashed
    \end{description}
  \item[cputime] How much CPU time (in seconds) the domain has used so
    far.
  \item[console] TCP port accepting connections to the domain's
    console.
  \end{description}
\end{quote}

The \path{xm list} command also supports a long output format when the
\path{-l} switch is used.  This outputs the fulls details of the
running domains in \xend's SXP configuration format.

For example, suppose the system is running the ttylinux domain as
described earlier.  The list command should produce output somewhat
like the following:
\begin{verbatim}
# xm list
Name              Id  Mem(MB)  CPU  State  Time(s)  Console
Domain-0           0      251    0  r----    172.2        
ttylinux           5       63    0  -b---      3.0    9605
\end{verbatim}

Here we can see the details for the ttylinux domain, as well as for
domain~0 (which, of course, is always running).  Note that the console
port for the ttylinux domain is 9605.  This can be connected to by TCP
using a terminal program (e.g. \path{telnet} or, better,
\path{xencons}).  The simplest way to connect is to use the
\path{xm~console} command, specifying the domain name or ID.  To
connect to the console of the ttylinux domain, we could use any of the
following:
\begin{verbatim}
# xm console ttylinux
# xm console 5
# xencons localhost 9605
\end{verbatim}

\section{xenstored}

Placeholder.
