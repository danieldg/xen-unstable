\chapter{Installing Xen on Red~Hat or Fedora~Core}

\section{Tips}
Here are a few pointers about using Xen / XenLinux on a Red~Hat or
Fedora~Core distribution:

\begin{enumerate}
\item Note that, because domains greater than~0 don't have any
  privileged access at all, certain commands in the default boot
  sequence will fail e.g.\ attempts to update the hwclock, change the
  console font, update the keytable map, start apmd (power management),
  or gpm (mouse cursor). Either ignore the errors (they should be
  harmless), or remove them from the startup scripts. Deleting the
  following links are a good start: {\path{S24pcmcia}},
  {\path{S09isdn}}, {\path{S17keytable}}, {\path{S26apmd}},
  {\path{S85gpm}}.

\item If you want to use a single root file system that works cleanly
  for both domain~0 and unprivileged domains, a useful trick is to use
  different `init' run levels. For example, use run level 3 for
  domain~0, and run level 4 for other domains. This enables different
  startup scripts to be run in depending on the run level number passed
  on the kernel command line.

\item If using NFS root files systems mounted either from an external
  server or from domain0 there are a couple of other gotchas. The
  default {\path{/etc/sysconfig/iptables}} rules block NFS, so part way
  through the boot sequence things will suddenly go dead.

\item If you're planning on having a separate NFS {\path{/usr}}
  partition, the RH9 boot scripts don't make life easy - they attempt to
  mount NFS file systems way to late in the boot process. The easiest
  way I found to do this was to have a {\path{/linuxrc}} script run
  ahead of {\path{/sbin/init}} that mounts {\path{/usr}}:

  \begin{quote}
    \begin{small}\begin{verbatim}
 #!/bin/bash
 /sbin/ipconfig lo 127.0.0.1
 /sbin/portmap
 /bin/mount /usr
 exec /sbin/init "$@" <>/dev/console 2>&1
\end{verbatim}\end{small}
  \end{quote}

%% $ XXX SMH: font lock fix :-)

  The one slight complication with the above is that
  {\path{/sbin/portmap}} is dynamically linked against
  {\path{/usr/lib/libwrap.so.0}} Since this is in {\path{/usr}}, it
  won't work. This can be solved by copying the file (and link) below
  the {\path{/usr}} mount point, and just let the file be `covered' when
  the mount happens.

\item In some installations, where a shared read-only {\path{/usr}} is
  being used, it may be desirable to move other large directories over
  into the read-only {\path{/usr}}. For example, you might replace
  {\path{/bin}}, {\path{/lib}} and {\path{/sbin}} with links into
  {\path{/usr/root/bin}}, {\path{/usr/root/lib}} and
  {\path{/usr/root/sbin}} respectively. This creates other problems for
  running the {\path{/linuxrc}} script, requiring bash, portmap, mount,
  ifconfig, and a handful of other shared libraries to be copied below
  the mount point --- a simple statically-linked C program would solve
  this problem.

\end{enumerate}